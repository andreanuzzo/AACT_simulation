\documentclass[]{article}
\usepackage{lmodern}
\usepackage{amssymb,amsmath}
\usepackage{ifxetex,ifluatex}
\usepackage{fixltx2e} % provides \textsubscript
\ifnum 0\ifxetex 1\fi\ifluatex 1\fi=0 % if pdftex
  \usepackage[T1]{fontenc}
  \usepackage[utf8]{inputenc}
\else % if luatex or xelatex
  \ifxetex
    \usepackage{mathspec}
  \else
    \usepackage{fontspec}
  \fi
  \defaultfontfeatures{Ligatures=TeX,Scale=MatchLowercase}
\fi
% use upquote if available, for straight quotes in verbatim environments
\IfFileExists{upquote.sty}{\usepackage{upquote}}{}
% use microtype if available
\IfFileExists{microtype.sty}{%
\usepackage{microtype}
\UseMicrotypeSet[protrusion]{basicmath} % disable protrusion for tt fonts
}{}
\usepackage[margin=1in]{geometry}
\usepackage{hyperref}
\hypersetup{unicode=true,
            pdftitle={Supplemental Material: Universal Shelter-in-place vs.~Advanced Automated Contact Tracing and Targeted Isolation: A Case for 21st-Century Technologies for COVID-19 and Future Pandemics},
            pdfauthor={Andrea Nuzzo, Can Ozan Tan, Ramesh Raskar, Rajiv Gupta},
            pdfborder={0 0 0},
            breaklinks=true}
\urlstyle{same}  % don't use monospace font for urls
\usepackage{longtable,booktabs}
\usepackage{graphicx,grffile}
\makeatletter
\def\maxwidth{\ifdim\Gin@nat@width>\linewidth\linewidth\else\Gin@nat@width\fi}
\def\maxheight{\ifdim\Gin@nat@height>\textheight\textheight\else\Gin@nat@height\fi}
\makeatother
% Scale images if necessary, so that they will not overflow the page
% margins by default, and it is still possible to overwrite the defaults
% using explicit options in \includegraphics[width, height, ...]{}
\setkeys{Gin}{width=\maxwidth,height=\maxheight,keepaspectratio}
\IfFileExists{parskip.sty}{%
\usepackage{parskip}
}{% else
\setlength{\parindent}{0pt}
\setlength{\parskip}{6pt plus 2pt minus 1pt}
}
\setlength{\emergencystretch}{3em}  % prevent overfull lines
\providecommand{\tightlist}{%
  \setlength{\itemsep}{0pt}\setlength{\parskip}{0pt}}
\setcounter{secnumdepth}{0}
% Redefines (sub)paragraphs to behave more like sections
\ifx\paragraph\undefined\else
\let\oldparagraph\paragraph
\renewcommand{\paragraph}[1]{\oldparagraph{#1}\mbox{}}
\fi
\ifx\subparagraph\undefined\else
\let\oldsubparagraph\subparagraph
\renewcommand{\subparagraph}[1]{\oldsubparagraph{#1}\mbox{}}
\fi

%%% Use protect on footnotes to avoid problems with footnotes in titles
\let\rmarkdownfootnote\footnote%
\def\footnote{\protect\rmarkdownfootnote}

%%% Change title format to be more compact
\usepackage{titling}

% Create subtitle command for use in maketitle
\providecommand{\subtitle}[1]{
  \posttitle{
    \begin{center}\large#1\end{center}
    }
}

\setlength{\droptitle}{-2em}

  \title{Supplemental Material: Universal Shelter-in-place vs.~Advanced
Automated Contact Tracing and Targeted Isolation: A Case for
21st-Century Technologies for COVID-19 and Future Pandemics}
    \pretitle{\vspace{\droptitle}\centering\huge}
  \posttitle{\par}
    \author{Andrea Nuzzo, Can Ozan Tan, Ramesh Raskar, Rajiv Gupta}
    \preauthor{\centering\large\emph}
  \postauthor{\par}
      \predate{\centering\large\emph}
  \postdate{\par}
    \date{2020-03-31}

\usepackage{booktabs}
\usepackage{longtable}
\usepackage{array}
\usepackage{multirow}
\usepackage{wrapfig}
\usepackage{float}
\usepackage{colortbl}
\usepackage{pdflscape}
\usepackage{tabu}
\usepackage{threeparttable}
\usepackage{threeparttablex}
\usepackage[normalem]{ulem}
\usepackage{makecell}
\usepackage{xcolor}

\renewcommand{\figurename}{Supplemental Figure}

\begin{document}
\maketitle

\hypertarget{data}{%
\section{Data}\label{data}}

Data are collected from the GitHub repo of the
\href{https://doi.org/10.1016/S1473-3099(20)30120-1}{JHU CSSE COVID-19
team} on 2020-03-31

\begin{figure}
\centering
\includegraphics{./SafePaths_WhitePaper_SupplMat_files/figure-latex/unnamed-chunk-2-1.eps}
\caption{Overview of cases, recoveries and fatalities in the US
population during the CIVID-19 pandemic as of \texttt{r\ Sys.Date()}}
\end{figure}

\hypertarget{approach}{%
\section{Approach}\label{approach}}

In this framework we analyze two possibilities to implement non-clinical
procedures to stop the spread of the epidemic:

\begin{itemize}
\item
  Advanced contact tracing: Through AACT, it is possible to inform
  Exposed (asymptomatic/non-infected) members of the community of the
  exposure risk. Once warned, they would ideally self-isolate themselves
  and prevent second-order spreading of the contagion. Therefore,
  self-isolated contacts will depend on the AACT penetrance \emph{p} in
  both the Infected and the Exposed population.
\item
  We are assuming efficacy 100\% (or rather, traced contacts receiving
  warnings and not self-isolating would pose as much risk as non-traced
  contacts). Self-isolated members might still develop symptoms.
\item
  The percentage of AACT penetration will also limit the further
  exposure, thus reducing the \(\beta\) transition between Susceptible
  and Exposed.
\item
  Traditional measures: in order to stop the contagion, authorities
  might recur to enforce social distancing through different measures,
  going from limitation of public gathering to full lockdown. We use the
  variable \emph{g} to model these interventions which will act
  aspecifically on Susceptible, Exposed and Infected population. This
  measure does not depend on the percentage of Infected patients, but
  will still limit the \(\beta\) of the Susceptible population.
  Quarantine will last for a time of 40 days (assumed reasonable in the
  current scenario)
\end{itemize}

\hypertarget{initial-values}{%
\section{Initial values}\label{initial-values}}

We assume the following initial parameters:

\begin{itemize}
\item
  \(T_{inf}\) = Duration of the \_infectious\_period (\textasciitilde2.9
  days, from literature)
\item
  \(T_{lat}\) = Latency period before development of symptoms
  (\textasciitilde{} 5.2, averaged from
  \href{https://www.eurosurveillance.org/content/10.2807/1560-7917.ES.2020.25.5.2000062}{literature})
\item
  Basic \(R_0\) = 3.02 from
  \href{https://doi.org/10.1016/S2214-109X\%2820\%2930113-3}{}
\end{itemize}

Preliminary death rate \(\mu = 0.057\) (1\% estimated by
\href{https://doi.org/10.1016/S1473-3099\%2820\%2930195-X}{Baud et
al.~2020} in the early phase of the disease)

\(\beta\) imputed from the definition of
\(\displaystyle R_{0}= \frac{\beta}{\gamma}\)

\newpage

\hypertarget{aact-model}{%
\section{AACT model}\label{aact-model}}

\begin{longtable}[]{@{}ll@{}}
\toprule
\begin{minipage}[b]{0.11\columnwidth}\raggedright
Compartment\strut
\end{minipage} & \begin{minipage}[b]{0.83\columnwidth}\raggedright
Functional definition\strut
\end{minipage}\tabularnewline
\midrule
\endhead
\begin{minipage}[t]{0.11\columnwidth}\raggedright
S\strut
\end{minipage} & \begin{minipage}[t]{0.83\columnwidth}\raggedright
Susceptible individuals\strut
\end{minipage}\tabularnewline
\begin{minipage}[t]{0.11\columnwidth}\raggedright
E\strut
\end{minipage} & \begin{minipage}[t]{0.83\columnwidth}\raggedright
Exposed to infection, unclear symptomatic conditions, potentially
infectious\strut
\end{minipage}\tabularnewline
\begin{minipage}[t]{0.11\columnwidth}\raggedright
I\strut
\end{minipage} & \begin{minipage}[t]{0.83\columnwidth}\raggedright
Infected, confirmed symptomatic and infectious\strut
\end{minipage}\tabularnewline
\begin{minipage}[t]{0.11\columnwidth}\raggedright
Sq\strut
\end{minipage} & \begin{minipage}[t]{0.83\columnwidth}\raggedright
Traced contacts, thus exposed but (self-)isolated\strut
\end{minipage}\tabularnewline
\begin{minipage}[t]{0.11\columnwidth}\raggedright
R\strut
\end{minipage} & \begin{minipage}[t]{0.83\columnwidth}\raggedright
Recovered, immune from further infection\strut
\end{minipage}\tabularnewline
\begin{minipage}[t]{0.11\columnwidth}\raggedright
D\strut
\end{minipage} & \begin{minipage}[t]{0.83\columnwidth}\raggedright
Case fatality (death due to COVID-19, not other causes)\strut
\end{minipage}\tabularnewline
\bottomrule
\end{longtable}

Here we will consider \(p\) as the percentage of adoption of the contact
tracing digital solution among the \emph{whole} population. We are
assuming that percentage of adoption corresponds to efficacy and
tempestivity of use. Moreover, we do not model the second and
third-grade exposure risks from the first contacts for simplicity.

\(\displaystyle \delta = \frac{1}{T_{lat}}\)

\(\displaystyle \frac{dS}{dt} = - (1-p)\frac{\beta}{N}SI\)

\(\displaystyle \frac{dE}{dt} = (1-p)\frac{\beta}{N}SI- \delta E - pE\)

\(\displaystyle \frac{dSq}{dt} = pE - \delta Sq\)

\(\displaystyle \frac{dI}{dt} = \delta E - \gamma I - \mu I + \delta Sq\)

\(\displaystyle \frac{dR}{dt} = \gamma I\)

\(\displaystyle \frac{dD}{dt} = \mu I\)

\begin{figure}
\centering
\includegraphics{SafePaths_WhitePaper_SupplMat_files/figure-latex/unnamed-chunk-5-1.eps}
\caption{Overview of the model flows used to simulate the contact
tracing and self-isolation non-clinical procedure to curb the spread of
COVID-19}
\end{figure}

\begin{figure}
\centering
\includegraphics{SafePaths_WhitePaper_SupplMat_files/figure-latex/unnamed-chunk-8-1.eps}
\caption{Change in distribution over time of the different
epidemiological compartments, a) Infected, b) Deaths and c)
Self-quarantined traced contacts depending on the increase of adoption
rate of the contact tracing digital solution. It is possible to see that
an adoption of 30\% would more than halven the occurrences of infected
cases and deaths. Dashed line represents the
\href{https://www.statnews.com/2020/03/10/simple-math-alarming-answers-covid-19/}{estimated
maximum ICU burden tolerable} during the COVID-19 epidemic.}
\end{figure}

\begin{longtable}[]{@{}rlll@{}}
\caption{Estimated decrease of peak disease impact indicators, thanks to
the deployment of platform for capillar contact tracing during the
COVID-19 epidemic in the United States}\tabularnewline
\toprule
adoption & max\_infected & max\_death & max\_Sq\tabularnewline
\midrule
\endfirsthead
\toprule
adoption & max\_infected & max\_death & max\_Sq\tabularnewline
\midrule
\endhead
0.0 & 25 987k & 42 314k & 0k\tabularnewline
0.1 & 22 000k & 40 629k & 15 448k\tabularnewline
0.2 & 17 591k & 38 185k & 18 722k\tabularnewline
0.3 & 12 808k & 34 535k & 16 361k\tabularnewline
0.4 & 7 782k & 28 872k & 11 016k\tabularnewline
0.5 & 3 028k & 18 895k & 4 575k\tabularnewline
0.6 & 31k & 298k & 49k\tabularnewline
0.7 & 8k & 17k & 12k\tabularnewline
0.8 & 7k & 8k & 12k\tabularnewline
0.9 & 7k & 5k & 12k\tabularnewline
\bottomrule
\end{longtable}

\newpage

\hypertarget{universal-shelter-in-place-model}{%
\section{Universal Shelter-in-place
model}\label{universal-shelter-in-place-model}}

\begin{figure}
\centering
\includegraphics{SafePaths_WhitePaper_SupplMat_files/figure-latex/unnamed-chunk-10-1.eps}
\caption{Overview of the model flows used to simulate the
government-imposed non-clinical procedure to curb the spread of
COVID-19}
\end{figure}

\begin{longtable}[]{@{}ll@{}}
\toprule
\begin{minipage}[b]{0.11\columnwidth}\raggedright
Compartment\strut
\end{minipage} & \begin{minipage}[b]{0.83\columnwidth}\raggedright
Functional definition\strut
\end{minipage}\tabularnewline
\midrule
\endhead
\begin{minipage}[t]{0.11\columnwidth}\raggedright
S\strut
\end{minipage} & \begin{minipage}[t]{0.83\columnwidth}\raggedright
Susceptible individuals\strut
\end{minipage}\tabularnewline
\begin{minipage}[t]{0.11\columnwidth}\raggedright
E\strut
\end{minipage} & \begin{minipage}[t]{0.83\columnwidth}\raggedright
Exposed to infection, unclear symptomatic conditions, potentially
infectious\strut
\end{minipage}\tabularnewline
\begin{minipage}[t]{0.11\columnwidth}\raggedright
I\strut
\end{minipage} & \begin{minipage}[t]{0.83\columnwidth}\raggedright
Infected, confirmed symptomatic and infectious\strut
\end{minipage}\tabularnewline
\begin{minipage}[t]{0.11\columnwidth}\raggedright
Q\strut
\end{minipage} & \begin{minipage}[t]{0.83\columnwidth}\raggedright
Isolated from the rest of the population through forced measures.
Unclear clinical definition\strut
\end{minipage}\tabularnewline
\begin{minipage}[t]{0.11\columnwidth}\raggedright
R\strut
\end{minipage} & \begin{minipage}[t]{0.83\columnwidth}\raggedright
Recovered, immune from further infection\strut
\end{minipage}\tabularnewline
\begin{minipage}[t]{0.11\columnwidth}\raggedright
D\strut
\end{minipage} & \begin{minipage}[t]{0.83\columnwidth}\raggedright
Case fatality (death due to COVID-19, not other causes)\strut
\end{minipage}\tabularnewline
\bottomrule
\end{longtable}

Here we will consider \(g\) as the strength of intervention, hard to
quantify numerically, but can be assumed to increase from limiting big
gathering events up to full lockdown, and \(\theta\) as the rate of
intervention (assumint time of intervention of half the current observed
time, 40 days). Here \(g\) will have effect on the Susceptible
population. Quarantined people will decrease after the intervetion time
(and ideally assigned to the Recovered, not the Susceptible population
for simplicity purposes). The incidence of intervention does \emph{not}
depend on the I compartment.

\(\displaystyle \theta = \frac{1}{T_{intervention}}\)

\(\displaystyle \frac{dS}{dt} = - (1-g)\frac{\beta}{N}SI - g\theta S\)

\(\displaystyle \frac{dQ}{dt} = g\theta S - \theta Q\)

\(\displaystyle \frac{dE}{dt} = (1-g)\frac{\beta}{N}SI - \delta E\)

\(\displaystyle \frac{dI}{dt} = \delta E - \gamma I - gI\)

\(\displaystyle \frac{dR}{dt} = \gamma I + \theta Q\)

\(\displaystyle \frac{dD}{dt} = \mu I\)

\begin{figure}
\centering
\includegraphics{SafePaths_WhitePaper_SupplMat_files/figure-latex/unnamed-chunk-13-1.eps}
\caption{Change in distribution over time of the different
epidemiological compartments, a) Infected, b) Exposed and c) Quarantined
individuals depending on the increase of government intervention. The
curves show that slightly stronger intervention measures are necessary
to reach similar decrease in incidence across the population but
impacting much more people, as they are applied aspecifically. Dashed
line represents the
\href{https://www.statnews.com/2020/03/10/simple-math-alarming-answers-covid-19/}{estimated
maximum ICU burden tolerable} during the COVID-19 epidemic}
\end{figure}

\begin{longtable}[]{@{}llll@{}}
\caption{Estimated decrease of peak disease impact indicators after
imposing external non-clinical intervention on the population to limit
the spread of Covid-19}\tabularnewline
\toprule
government\_intervention & max\_infected & max\_death &
max\_q\tabularnewline
\midrule
\endfirsthead
\toprule
government\_intervention & max\_infected & max\_death &
max\_q\tabularnewline
\midrule
\endhead
0k & 25 987k & 42 314k & 0k\tabularnewline
0k & 14 259k & 31 106k & 22 059k\tabularnewline
0k & 3 833k & 13 526k & 42 592k\tabularnewline
0k & 240k & 1 068k & 58 757k\tabularnewline
0k & 32k & 124k & 71 555k\tabularnewline
0k & 12k & 35k & 82 447k\tabularnewline
0k & 8k & 16k & 91 991k\tabularnewline
0k & 8k & 10k & 100 475k\tabularnewline
0k & 7k & 6k & 108 112k\tabularnewline
0k & 7k & 5k & 115 048k\tabularnewline
\bottomrule
\end{longtable}


\end{document}
